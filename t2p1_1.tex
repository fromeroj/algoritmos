\documentclass[9pt,spanish]{article}
      \usepackage[spanish,activeacute]{babel}
      \usepackage{anysize}
        %para tamaño carta, para otros eligieran la correcta
	  \papersize{27.9cm}{21.5cm} 
		%\marginsize{Izque}{Derec}{Arrib}{Abajo}	
	  \marginsize{2.0cm}{2.0cm}{1.0cm}{1.0cm}
	  \usepackage[utf8]{inputenc}
      \usepackage{enumerate}
      \usepackage{algpseudocode}
      \usepackage{algorithm}
	  %\usepackage{algorithmic}
	  \usepackage{amsmath}
	  \usepackage{titlesec}
      \usepackage{dsfont}
       \usepackage{tikz}
	  \usetikzlibrary{automata,positioning}
	  \usepackage[matrix,arrow]{xy}
	  \usetikzlibrary{intersections}
	  \usepackage{tcolorbox}
      \parindent 0em
      \parskip 1ex
      \title{Algoritmos\\ \large{TAREA 2} }
      \author{Compilado de respuestas}
\begin{document}
    \maketitle  
      \newtheorem{teo}{Teorema}[section]    
%****************Formato de las secciones
\titleformat{\subsection}[frame]
{\normalfont}{\filright\Large
\ Problema \thesubsection \ }
{6pt}{\bfseries\filright}
%*****************************************


\section{Teorema de Landau}
    
  \subsection{Demuestre en detalle que si una secuencia es de puntajes de un torneo, entonces satisface la condici\'on.}
  
  
  Se demostrar\'a por inducci\'on sobre el tama\~no de la secuencia.\\
\begin{itemize}
	\item CASO BASE: \\
	Para un torneo de tama\~no 2 se tiene la unica secuencia $(0,1)$ que cumple con el teorema.\\
	Para un torneo de tama\~no n=3 se tienen las secuencias $(0,1,2)$ y $(1,1,1)$, las cuales cumplen la condici\'on. \\
	\item HIP\'OTESIS: \\ Si $s(s_1 \leq s_2 \leq ... \leq s_n)$ es una secuencia de tama\~no $n$ correspondiente a un torneo, se cumple que: $\displaystyle\sum_{i=1}^k s_i \geq {k \choose 2}$ e igual para $k=n$ 
	\item P.D. para una secuencia de tama\~no n+1 \\
	A partir de un torneo de tama\~no $n$ construimos unmo de tama\~no $n+1$ a\~nadiendo un elemento que se relacionar\'a con los $n$ elementos iniciales, resultando entonces que el torneo de tama\~no $n$ es un subtorneo del que se acaba de contruir (tama\~no $n+1$), cuya secuencia se contruye de la siguiente manera:\\
	Suponemos que hay $r$ elementos del subtorneo de tama\~no $n$ cuya arista llega hacia el nuevo elemento agregado, donde $0 \leq r \leq n$ (seria cero si no hay ninguna arista hacia el nuevo elemento  $y$ n si todas las aristas mencionadas llegan al nuevo). Esto hace que dichos elementos incrementen en uno su marcador, por lo tanto, la nueva secuencia de marcadores para el subtorneo de tama\~no $n$ es:
	 $\displaystyle\sum_{i=1}^n {s_i} + r$ \\
	 y el marcador del nuevo elemento es $n-r$.
	 Como se puede observar, ahora la secuencia  $s(s_1 \leq s_2 \leq ... \leq s_n)$ para el torneo de taman\~no $n+1$ es mayor (cuando $r>0$) o igual (cuando $r=0$) a ${n \choose 2}$ por la hip\'otesis.
	 Ahora,   $\displaystyle\sum_{i=1}^{n+1} s_i = \displaystyle\sum_{i=1}^n {s_i} + r + n - r$, por lo tanto:
	  \begin{eqnarray}
	  \label{eqn:smasuno}
	  \displaystyle\sum_{i=1}^{n+1} s_i &=& \displaystyle\sum_{i=1}^n {s_i} +  n
	  \end{eqnarray}
Pero:
	  \begin{eqnarray}
	  \displaystyle\sum_{i=1}^{n} s_i &\geq& {n \choose 2} \nonumber
	  \end{eqnarray}
Sumando $n$ de ambos lados:
	  \begin{eqnarray}
	  \displaystyle\sum_{i=1}^{n} s_i +n  &\geq& {n \choose 2} +n  \nonumber \\
	  \displaystyle\sum_{i=1}^{n} s_i +n  &\geq& \displaystyle\frac{n(n-1)}{2} +n  \nonumber \\
	  \displaystyle\sum_{i=1}^{n} s_i +n  &\geq& \displaystyle\frac{n(n-1+2)}{2} \nonumber \\
	  \displaystyle\sum_{i=1}^{n} s_i +n  &\geq& \displaystyle\frac{n(n+1)}{2} \nonumber  \\
	  \label{eqn:snueva}
	  \displaystyle\sum_{i=1}^{n} s_i +n &\geq& {n+1 \choose 2} 
	  \end{eqnarray}
Sustituyendo \ref{eqn:snueva} en \ref{eqn:smasuno}:
\begin{eqnarray}
	  \displaystyle\sum_{i=1}^{n+1} s_i &\geq& {n+1 \choose 2} 
	  \end{eqnarray}
Que es nuestra hip\'otesis de inducci\'on. $\surd$
	
\end{itemize}  
Observaciones: El ayudante comentó que era mas fácil quitar uno que añadirlo.

Al finaL comenta que se demuestra la igualdad pero no queda muy clara la desigualdad.

CALIFICACIÓN: 0.95




\end{document}
