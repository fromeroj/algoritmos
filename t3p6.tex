\documentclass[9pt,spanish]{article}
      \usepackage[spanish,activeacute]{babel}
      \usepackage{anysize}
        %para tamaño carta, para otros eligieran la correcta
	  \papersize{27.9cm}{21.5cm} 
		%\marginsize{Izque}{Derec}{Arrib}{Abajo}	
	  \marginsize{2.0cm}{2.0cm}{1.0cm}{1.0cm}
	  \usepackage[utf8]{inputenc}
      \usepackage{enumerate}
      \usepackage{algpseudocode}
      \usepackage{algorithm}
	  %\usepackage{algorithmic}
	  \usepackage{amsmath}
	  \usepackage{titlesec}
      \usepackage{dsfont}
       \usepackage{tikz}
	  \usetikzlibrary{automata,positioning}
	  \usepackage[matrix,arrow]{xy}
	  \usetikzlibrary{intersections}
	  \usepackage{tcolorbox}
      \parindent 0em
      \parskip 1ex
      \title{Algoritmos\\ \large{TAREA 3} }
      
      \author{Compilación}
      \numberwithin{equation}{section}
\begin{document}
    \maketitle  
%****************Formato de las secciones
\titleformat{\section}[frame]
{\normalfont}{\filright\Large
\ Problema \thesection \ }
{6pt}{\bfseries\filright}
%*****************************************
\section*{Demuestra que cuando todos los pesos en las aristas de una gr\'afica son distintos, el \'arbol generador de peso m\'inimo es \'unico.  Da un ejemplo general donde haya aristas con pesos iguales y  que tenga mas de un \'arbol generador de peso m\'inimo.  Da un ejemplo general de una gr\'afica donde haya aristas con pesos iguales, y que tenga un \'unico arbol generador de peso m\'inimo. }

Se demostrará por contradicción:

Supongo que tengo dos árboles generadores de peso mínimo T con aristas ordenadas de menor a mayor peso ($e_1,e_2.. e_k$) y T' con aristas ordenadas de menor a mayor peso ($e'_1,e'_2.. e'_k$).  El peso de T es igual al peso de  T', es decir, w(T)=w(T').

Sea $e_p$ la primera arista tal que w($e_i$) $\neq$ w($e_i'$). 

Sea S=$T' \cup \{e_p\}$. Esto genera un ciclo llamado $C_1'$, ya que $T'$ es un árbol generador mínimo.  Existe $e_q' \in C_1'$ tal que $e_q' \not\in T$. w($e_p$) $\neq$ w($e_q'$) 

Suponemos un corte en T', en el cual $e_p$ y $e_q'$ son puentes entre las particiones de T'.  Si quitamos el peso de  $e_q'$ al de T' y sumamos el peso de $e_p$, no debe cambiar el peso de T', lo que implica que w($e_p$)=w($e_q'$), lo cual es una contradicción ya que se dijo que los pesos de ambas aristas deben ser diferentes, por lo tanto, el supuesto de que tengo dos árboles generadores mínimos es falso.

Es decir, el árbol generador mínimo con aristas de distinto peso es único. 



\end{document} 	
