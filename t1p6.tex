\documentclass[12pt]{article}
\usepackage[utf8]{inputenc}
\usepackage{listings}
\usepackage{ amssymb }
\usepackage{color}
\begin{document}
\lstset{language=python}
\begin{itemize}

\item[\bf{Pregunta 6}]El algoritmo de BFS visto en clase utiliza una cola de vértices, y cada vértice tiene asignado un color (gris,blanco,negro), y un estimado de distancia d. El estimado d(v) siempre es mayor o igual a la distancia del origen a v, y al final del algoritmo, d(v) es igual a la distancia del origen a v. Presenta y demuestra las invariantes que satisfacen los vértices en la cola, con respecto a d y a su color.

Por claridad cambiemos (gris,blanco,negro) por (encontrado,visitados y no visitado)
\begin{lstlisting}[frame=single] 
1 def BFS(G,v):
2    create a queue Encontrados
3    create a set Visitados
4    create queue Ordenados
5    enqueue v onto Encontrados
6    while Encontrados is not empty:
7        t <- Encontrados.dequeue()
8        for all vertex u in G.adjacentEdges(t) do
9             if u is not in Ordenados:
10                if u is not in Encontrados:
11                   enqueue u into Encontrados
12            add u into Ordenados
\end{lstlisting}

\item[invariantes]
  \begin{itemize}
 \item En cada iteración de $(6)$ el conjunto Ordenados aumenta en uno su tamaño.\\
   Lo cual es cierto pues $12$ es ejecutado incondicionalmente una vez cada que se pasa por $6$.
   \item Cada elemento entra una unica vez y sale una unica vez de Encontrados.\\
     Esto se debe a que $11$ es condicion de $9$ y de $10$, es decir, que se añade a Encontrados, si no esta en Encontrados o en Ordenados, es decir que se visita por primera vez, de ahi que solo entra una vez a Encontrados
   \end{itemize}

\end{itemize}
\end{document}
