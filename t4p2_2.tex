\documentclass[9pt,spanish]{article}
      \usepackage[spanish,activeacute]{babel}
      \usepackage{anysize}
        %para tamaño carta, para otros eligieran la correcta
	  \papersize{27.9cm}{21.5cm} 
		%\marginsize{Izque}{Derec}{Arrib}{Abajo}	
	  \marginsize{2.0cm}{2.0cm}{1.0cm}{1.0cm}
	  \usepackage[utf8]{inputenc}
      \usepackage{enumerate}
      \usepackage{algpseudocode}
      \usepackage{algorithm}
	  %\usepackage{algorithmic}
	  \usepackage{amsmath}
	  \usepackage{titlesec}
      \usepackage{dsfont}
       \usepackage{tikz}
	  \usetikzlibrary{automata,positioning}
	  \usepackage[matrix,arrow]{xy}
	  \usetikzlibrary{intersections}
	  \usepackage{tcolorbox}
      \parindent 0em
      \parskip 1ex
      \title{Algoritmos\\ \large{TAREA 4} }
      
      \author{Israel Sandoval Grajeda y Fausto Salazar Mora}
      \numberwithin{equation}{section}
\begin{document}
    \maketitle  
%****************Formato de las secciones
\titleformat{\section}[frame]
{\normalfont}{\filright\Large
\ Problema  }
{6pt}{\bfseries\filright}
%*****************************************

\section*{Divide y vencerás}
\subsection*{Algoritmo para la multiplicación de bits.}
Para la ejecución del algoritmo presentado a continuación se deben realizar las siguientes consideraciones:
\begin{enumerate}
\item a y b son arreglos de bits
\item Su longitud es $2^n$
\end{enumerate}
\begin{algorithm}[H]
	\caption{Multiplicación de bits}
	\begin{algorithmic}[1]
	\State MultBits(a,b)
	\State n=longitud(a)/2
	\If{n==1}
	\State return a*b
	\EndIf a1=a[:n]
	\State a2=a[n:]
	\State b1=b[:n]
	\State b2=b[n:]
	\State z2=MultBits(a2,b2)
	\State z0=MultBits(a1,b1)
	\State z1=MultBits(a1+a2,b1+b2)-z2-z0
	\State ShiftL(z2,n)
	\State ShiftL(z1,n/2)
	\State return z2+z1+z0
	\end{algorithmic}
\end{algorithm}

Complejidad:

En la función ShiftL hay un ciclo hasta n para hacer los corrimientos, las demás operaciones en esta función son constantes así pues el tiempo es lineal sobre el tamaño de n es decir O(n)

En la recursión todas las operaciones son constantes y la recursión tiene la forma de 3 T(n/2) entonces la complejidad de la recursión es T(n) = 3 T(n/2) + O(n).
Aplicando el método maestro tengo a=3 b=2 y f(n) = O(n) comparando f(n) con $n^{log_ba}$ cae en el caso 1 del método maestro y por lo tanto T(n) = O($n^{log_23}$)

Correctez:

El predicado siguiente indica que es correcto:\\
En cada llamada recursiva de la función, el tamaño de las entradas disminuye a la mitad y al volver de la recursión siempre me regresa el resultado de multiplicar las entradas.
\end{document} 	
