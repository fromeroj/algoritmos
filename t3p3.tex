\documentclass[9pt,spanish]{article}
      \usepackage[spanish,activeacute]{babel}
      \usepackage{anysize}
        %para tamaño carta, para otros eligieran la correcta
	  \papersize{27.9cm}{21.5cm} 
		%\marginsize{Izque}{Derec}{Arrib}{Abajo}	
	  \marginsize{2.0cm}{2.0cm}{1.0cm}{1.0cm}
	  \usepackage[utf8]{inputenc}
      \usepackage{enumerate}
      \usepackage{algpseudocode}
      \usepackage{algorithm}
	  %\usepackage{algorithmic}
	  \usepackage{amsmath}
	  \usepackage{titlesec}
      \usepackage{dsfont}
       \usepackage{tikz}
	  \usetikzlibrary{automata,positioning}
	  \usepackage[matrix,arrow]{xy}
	  \usetikzlibrary{intersections}
	  \usepackage{tcolorbox}
      \parindent 0em
      \parskip 1ex
      \title{Algoritmos\\ \large{TAREA 3} }
      
      \author{Compilación}
      \numberwithin{equation}{section}
\begin{document}
    \maketitle  
%****************Formato de las secciones
\titleformat{\section}[frame]
{\normalfont}{\filright\Large
\ Problema \thesection \ }
{6pt}{\bfseries\filright}
%*****************************************

\section*{Explica en detalle en que consiste el argumento de la prueba vista en clase para demostrar que el algoritmo de las minimas paradas para cargar gasolina es optimo.}

Este argumento se llama argumento de intercambio y consiste en que dada la solución de un algoritmo greedy llámese “solución greedy, abreviado SG” se supone que hay otra solución dada por otro algoritmo y que es óptima llámese “solución óptima, abreviado SO”. Hay que asumir que la solución óptima y la solución greedy no son iguales por lo que:
\begin{itemize}
\item Hay elementos en SG que no aparece en SO y elementos en SO que no aparece en SG \\
ó 
\item Hay elementos en SO que aparecen en orden diferente a como aparecen en SG (esto se conoce como inversiones).

\end{itemize}
Entonces la prueba se efectúa haciendo intercambios en los elementos de SO cuidando que no se empeore la calidad de la solución dada por la SO original y que con este intercambio la forma de SO se acerca más a la forma de SG de manera tal que si se repiten estos intercambios tantas veces como sea necesario terminamos sin diferencias entre SO y SG lo cual demuestra que ambos son óptimos.

El corazón del argumento es suponer una solución óptima lo mas parecida posible a la solución greedy,pero al analizar las diferencias, se encuentra que el algoritmo greedy proporcionó una solución mejor a la óptima.



\end{document} 	
