\documentclass[12pt]{article}
\usepackage[utf8]{inputenc}
\usepackage{listings}
\usepackage{ amssymb }
\usepackage{color}

\title{Estructura de datos y teoría de algoritmos\\
Tarea 1 }
\begin{document}
\lstset{language=python}
\maketitle

\section{Tarea 1 Problema1}
\begin{itemize}
  \item[\bf{Pregunta 1}] Demuestra que todo torneo $T$ tiene un rey y un súbdito. Es decir, un vértice desde el cual se puede llegar a todos los demás vértices pasando a lo más por dos arcos, y uno al cual todos pueden llegar pasando a lo más por dos arcos

  \item[Demostración:]
    Sea $T=(V,E)$ un torneo y $ v \in V $ tal que $|out(v)| \ge |out(u)| \forall u\in V $ es decir la cardinalidad de la exvecindad de $v$ es máxima en $T$\\
    denotaremos $x \rightarrow  y, si (x,y)\in E$\\
    Afirmación:\\
    $v$ es un rey en $T$\\
    Es decir $$\forall u\ne v \in V ,  (v\rightarrow u) \vee (\exists u_v \in V  |  (v \rightarrow u_v) \wedge (u_v \rightarrow u))$$
    Demostración por reducción al absurdo:\\
    Supongamos que no es cierto, es decir:
    $$\exists u\ne v \in V ,  (u \rightarrow v) \wedge (\forall u_v \in V  ,  (v \rightarrow u_v)  \Rightarrow (u \rightarrow u_v)  ).$$

  \item[Existencia del súbdito:]
    Para demostrar que cada torneo es un súbdito, basta notar que un rey en un torneo $G$ es un súbdito en el torneo $\widehat G$ que se obtiene cambiando la orientación de cada arista en $G$, así, por su significado simétrico, como cada torneo tiene un rey, cada torneo tiene un súbdito.\\

\end{itemize}
\end{document}
