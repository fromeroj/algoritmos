\documentclass[12pt]{article}
\usepackage[utf8]{inputenc}
\usepackage{listings}
\usepackage{ amssymb }
\usepackage{color}
\begin{document}
\lstset{language=python}

\begin{itemize}
\item[\bf{Pregunta 2}] Presenta un algoritmo para encontrar un rey y analiza su complejidad y correctez.

\item[Algoritmo]
  El algoritmo es el sigiente:
  Recorrase el torneo buscando el elemento que tiene exvecindad máxima.

\begin{lstlisting}[frame=single] 
  def busca_rey(T)
  maximo = 0
  rey = none
  for {v in T.vertices}:
     if len(v.adjacent) > maximo:
       maximo = len(v.adjacent)
       rey = v
  return v
\end{lstlisting}

Este algoritmo busca exhaustivamente sobre los vértices, por lo que necesariamente encuentra aquel de máxima exvecindad.\\
Asumiendo que la representación de $G$ es la  de adyacencia con listas ligadas, el algoritmo es del orden de $O(n + m)$, pues cada vértice se recorre una vez y ya en el y solo se toma el tamaño de la lista de adyacencia, lo cual es en el caso de una lista ligada $O(m)$.

\end{itemize}
\end{document}
