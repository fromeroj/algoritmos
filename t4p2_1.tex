\documentclass[9pt,spanish]{article}
      \usepackage[spanish,activeacute]{babel}
      \usepackage{anysize}
        %para tamaño carta, para otros eligieran la correcta
	  \papersize{27.9cm}{21.5cm} 
		%\marginsize{Izque}{Derec}{Arrib}{Abajo}	
	  \marginsize{2.0cm}{2.0cm}{1.0cm}{1.0cm}
	  \usepackage[utf8]{inputenc}
      \usepackage{enumerate}
      \usepackage{algpseudocode}
      \usepackage{algorithm}
	  %\usepackage{algorithmic}
	  \usepackage{amsmath}
	  \usepackage{titlesec}
      \usepackage{dsfont}
       \usepackage{tikz}
	  \usetikzlibrary{automata,positioning}
	  \usepackage[matrix,arrow]{xy}
	  \usetikzlibrary{intersections}
	  \usepackage{tcolorbox}
      \parindent 0em
      \parskip 1ex
      \title{Algoritmos\\ \large{TAREA 4} }
      
      \author{Israel Sandoval Grajeda y Fausto Salazar Mora}
      \numberwithin{equation}{section}
\begin{document}
    \maketitle  
%****************Formato de las secciones
\titleformat{\section}[frame]
{\normalfont}{\filright\Large
\ Problema  }
{6pt}{\bfseries\filright}
%*****************************************


\section*{Divide y vencer\'as}
\subsection*{Cierre convexo}

\begin{algorithm}[H]
	\caption{Cierre convexo}
	\begin{algorithmic}[1]
	\State Odena(S,x)
	\State cierreconvexo(S) //Regresa la secuencia de puntos del cierre convexo
	\State n=longitud(S)	   
	\If{len(L)$<$3}
	\State	return S  //Caso base
	\EndIf
	\State I=cierreconvexo(S[0..n/2$\rfloor$])
	\State D=cierreconvexo(S[n/2$\rfloor$+1..Tam(L)-1]) //
	\State return UnionConvexa(I,D) 
	\end{algorithmic}
\end{algorithm}

Ahora se presenta la funcion UnionConvexa()
\begin{algorithm}[H]
	\caption{Unión de las soluciones parciales del cierre convexo}
	\begin{algorithmic}[1]
	\State Union Convexa(LI,LD)
	\State  Codigo...
	\end{algorithmic}
\end{algorithm}
\begin{enumerate}
\item Análisis de complejidad\\
Tomando en cuenta lo que ya se reviso
\begin{itemize}
\item En el ordenamiento del inicio O(n log n) 1 sola vez
\item En el caso base tengo tiempo constante O(1)
\item En el caso recursivo la entrada se divide en 2 así que es 2 T(n/2).
\item Falta análisis de Unión convexa pero se supone que debe ser   O(n) para que el algoritmo tenga complejidad O(nlog(n)).
\item Por lo tanto el caso recursivo completo es T(n) = 2 T(n/2) + O(n)
Aplicando el método maestro a= 2 b=2 y f(n) =O(n)
Comparo f(n) con $n^{log_ba}$ puesto que $log_22 = 1$ tengo n = n que cae en el caso 2 del método maestro y T(n) = O(n log n). Finalmente tomando en cuenta también el ordenamiento O(n log n) + O (n log n) en total termino con O(n log n).
\end{itemize}
\item Análisis de correctez \\
El predicado siguiente indica que es correcto:
En cada llamada recursiva de la función el tamaño de la entrada disminuye a la mitad y al volver de la recursión siempre me regresa el cierre convexo de la entra que recibió.

\end{enumerate}

\newpage


\end{document} 	
