\documentclass[9pt,spanish]{article}
      \usepackage[spanish,activeacute]{babel}
      \usepackage{anysize}
        %para tamaño carta, para otros eligieran la correcta
	  \papersize{27.9cm}{21.5cm} 
		%\marginsize{Izque}{Derec}{Arrib}{Abajo}	
	  \marginsize{2.0cm}{2.0cm}{1.0cm}{1.0cm}
	  \usepackage[utf8]{inputenc}
      \usepackage{enumerate}
      \usepackage{algpseudocode}
      \usepackage{algorithm}
	  %\usepackage{algorithmic}
	  \usepackage{amsmath}
	  \usepackage{titlesec}
      \usepackage{dsfont}
       \usepackage{tikz}
	  \usetikzlibrary{automata,positioning}
	  \usepackage[matrix,arrow]{xy}
	  \usetikzlibrary{intersections}
	  \usepackage{tcolorbox}
      \parindent 0em
      \parskip 1ex
      \title{Algoritmos\\ \large{TAREA 2} }
      \author{Compilado de respuestas}
\begin{document}
    \maketitle  
      \newtheorem{teo}{Teorema}[section]    
%****************Formato de las secciones
\titleformat{\subsection}[frame]
{\normalfont}{\filright\Large
\ Problema \thesubsection \ }
{6pt}{\bfseries\filright}
%*****************************************


\section*{Dijkstra}
\subsection*{Demuestra porque el algoritmo de Dijsktra no funciona en gr\'aficas dirigidas con pesos negativos y da un ejemplo donde falla}

Antes de hacer la demostración  se propone un ejemplo donde el algoritmo de Dijsktra no funciona en grafo dirigido con un arco de peso negativo:

\begin{tikzpicture}
\path[yshift=1.5cm,shape=circle]
(0,1) node [style=draw, accepting](a1){A} (2,2) node [style=draw] (a2){B}
(2,0)node [style=draw](a3){C};
\draw[->] (a1) -- (a2) node [above,midway] {$5$};
\draw[->] (a2) -- (a3) node [above,midway] {$-10$};
\draw[->] (a3) -- (a1) node [above,midway] {$2$};

\end{tikzpicture}

Donde el vértice A es el origen.

De acuerdo al algoritmo de Dijsktra:
\begin{enumerate}
\item Se toma el vértice A y se le asigna distancia a cumulada igual a cero y se marca como un nodo permanente.
\item Se elige al siguiente de menort distancia que salga de A, es decir, al nodo ''C'' y se marca como permanente (cabe señalar que una vez marcado no se puede modificar su distancia, que en este caso es 2.
\item La ruta final es ''A-C'' con p=2
\end{enumerate}

Sin embargo, la ruta elegida no es la correcta, ya que la ruta ''A-B-C'' tiene peso p=-5, por lo tanto el algoritmo no funcona para este tipo de gráficas.


En general el algoritmo de Dijsktra va construyendo el camino mas corto conforme avanza a trav\'es de la gr\'afica, lo cual tiene dos implicaciones:
\begin{enumerate}
\item Cada vez que hace una iteraci\'on define el camino mas corto para cierta cantidad de nodos y no puede regresar por ese camino para cambiar algo.
\item No puede predecir cuando una ruta tiene un tramo con un peso negativo, sobre todo si se encuentra despu\'es de un tramo con peso muy grande.
\end{enumerate}
Nota: Los puntos anteriores los cuestionó mucho el ayudante pero los incluyo porque en el problema se obtuvo 1 como calificación.  Bastará ser mas precisos en las definiciones.

Para demostrar que el algoritmo de Dijsktra no funciona con pesos negativos, supongamos una gr\'afica de tama\~no $n$ donde tengo un camino:
\begin{eqnarray}
\label{eqn:caminox}
S   \xrightarrow{x_0} v_1 \xrightarrow{x_1} \cdots \xrightarrow{x_{m-1}} v_m \xrightarrow{x_m} T 
\end{eqnarray}
 y otro 
\begin{eqnarray}
\label{eqn:caminoy}
S \xrightarrow{y} u \xrightarrow{z} T 
\end{eqnarray} 
 Tal que
\begin{itemize}
\item   $\displaystyle\sum_{i=1}^{m} x_i < y$
\item  $z<0$ 
\item  $\displaystyle\sum_{i=1}^{m} x_i > y+z$ 
\end{itemize}
Lo cual quiere decir que  el camino \ref{eqn:caminoy}  es menor al \ref{eqn:caminox}.
Sin embargo, al ejecutar el algoritmo, siempre se elegir\'an los $x_i$  y al final a $y$, por lo tanto, el camino trazado por el algoritmo ser\'a a trav\'es de las $x_i$ el cual no es el mas corto, llegando a una contradicci\'on.  Por lo tanto, el algoritmo de Dijsktra no funciona para pesos negativos. $\surd$ \\
Ejemplo general:

\begin{tikzpicture}
%[every node/.style={draw}]

\path[yshift=1.5cm,shape=circle]
(0,1) node [style=draw](a1){S} (2,2) node [style=draw] (a2){$v_1$}
(4,2) node [style=draw] (a3){$v_2$} (6,2) node [style=draw] (a4){$v_3$} (8,1)node [style=draw](a5){T} (4,0)node [style=draw](a6){u};
\draw[->] (a1) -- (a2) node [above,midway] {$2$};
\draw[->] (a2) -- (a3) node [above,midway] {$4$};
\draw[->] (a3) -- (a4) node [above,midway] {$6$};
\draw[->] (a4) -- (a5) node [above,midway] {$3$};
\draw[->] (a1) -- (a6) node [above,midway] {$16$};
\draw[->] (a6) -- (a5) node [above,midway] {$-10$};
% -- (a4) -- (a5) -- (a6) --  -- (a1);
\end{tikzpicture}


\end{document}
