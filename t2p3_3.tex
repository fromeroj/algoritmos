\documentclass[9pt,spanish]{article}
      \usepackage[spanish,activeacute]{babel}
      \usepackage{anysize}
        %para tamaño carta, para otros eligieran la correcta
	  \papersize{27.9cm}{21.5cm} 
		%\marginsize{Izque}{Derec}{Arrib}{Abajo}	
	  \marginsize{2.0cm}{2.0cm}{1.0cm}{1.0cm}
	  \usepackage[utf8]{inputenc}
      \usepackage{enumerate}
      \usepackage{algpseudocode}
      \usepackage{algorithm}
	  %\usepackage{algorithmic}
	  \usepackage{amsmath}
	  \usepackage{titlesec}
      \usepackage{dsfont}
       \usepackage{tikz}
	  \usetikzlibrary{automata,positioning}
	  \usepackage[matrix,arrow]{xy}
	  \usetikzlibrary{intersections}
	  \usepackage{tcolorbox}
      \parindent 0em
      \parskip 1ex
      \title{Algoritmos\\ \large{TAREA 2} }
      \author{Compilado de respuestas}
\begin{document}
    \maketitle  
      \newtheorem{teo}{Teorema}[section]    
%****************Formato de las secciones
\titleformat{\subsection}[frame]
{\normalfont}{\filright\Large
\ Problema \thesubsection \ }
{6pt}{\bfseries\filright}
%*****************************************


\section*{Tema Ford}
\subsection*{c. Demuestra porque en la versi\'on avanzada, la complejidad es de orden $n \times m$.}
Algoritmo de Ford Avanzado

\begin{algorithm}[H]
	\caption{Algoritmo de Ford Avanzado}
	\begin{algorithmic}[1]
	\For{ every $v \in V$}
	\State $\lambda(v) \leftarrow \infty$
	\EndFor
	\State $\lambda(s) \leftarrow 0$
	\Repeat
	\State Flag $\leftarrow $ false
	\For{every $1 \leq i \leq m$}
	\State let $u \leftarrow v$
	\If{$\lambda(v)$ is finite and $\lambda(v)>\lambda(u) + l(ei)$}
	\State  $\lambda(v) \leftarrow \lambda(u)+l(ei)$
	\State Flag $\leftarrow$ True
	\EndIf
	\EndFor
	\Until Flag=False
	
	\end{algorithmic}
\end{algorithm}

Si el grafo no tiene ciclos negativos, el algoritmo avanzado de Ford termina en O(m,n).

Demostración: 
Si v es accesible desde s y ya que no hay ciclos negativos accesibles, una ruta mínima de s a v es simple y consta de n-1 aristas menos.
Por lo que durante la n-ésima aplicación del loop (lineas 7-13) ningún vértice mejora su valor y el procedimiento termina ya que la complejidad del loop (lineas 7-13) es O(m), el procedimiento completo toma O(m+n) tiempo.


  
  

\end{document}
