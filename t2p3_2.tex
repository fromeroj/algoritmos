\documentclass[9pt,spanish]{article}
      \usepackage[spanish,activeacute]{babel}
      \usepackage{anysize}
        %para tamaño carta, para otros eligieran la correcta
	  \papersize{27.9cm}{21.5cm} 
		%\marginsize{Izque}{Derec}{Arrib}{Abajo}	
	  \marginsize{2.0cm}{2.0cm}{1.0cm}{1.0cm}
	  \usepackage[utf8]{inputenc}
      \usepackage{enumerate}
      \usepackage{algpseudocode}
      \usepackage{algorithm}
	  %\usepackage{algorithmic}
	  \usepackage{amsmath}
	  \usepackage{titlesec}
      \usepackage{dsfont}
       \usepackage{tikz}
	  \usetikzlibrary{automata,positioning}
	  \usepackage[matrix,arrow]{xy}
	  \usetikzlibrary{intersections}
	  \usepackage{tcolorbox}
      \parindent 0em
      \parskip 1ex
      \title{Algoritmos\\ \large{TAREA 2} }
      \author{Compilado de respuestas}
\begin{document}
    \maketitle  
      \newtheorem{teo}{Teorema}[section]    
%****************Formato de las secciones
\titleformat{\subsection}[frame]
{\normalfont}{\filright\Large
\ Problema \thesubsection \ }
{6pt}{\bfseries\filright}
%*****************************************


\section*{Tema Ford}
\subsection*{b Explica porque el algoritmo es de complejidad exponencial}
Se debe al hecho de que no hay una elección determinada de la arista a examinar en el ciclo. Esto puede generar que se examinen en un orden exponencial.

\end{document}
